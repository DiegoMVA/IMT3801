\documentclass{article}
\usepackage[utf8]{inputenc}
\usepackage{amsmath, amsthm, amssymb, mathpazo, isomath, mathtools}
\usepackage{subcaption,graphicx,pgfplots}
\usepackage{fullpage}
\usepackage{booktabs}
\usepackage{hyperref}
\usepackage{algorithm, algorithmic}
\usepackage{mathtools}
\usepackage{todonotes}

\newcommand{\example}[1]{\todo[inline,color=green!30!white]{\textbf{Example:} #1}}

\title{Homework 2}
\author{Nicol\'as A Barnafi}
%\date{}

\renewcommand{\P}{{\mathbb{P}}}
\renewcommand{\vec}{\vectorsym}
\newcommand{\mat}{\matrixsym}
\newcommand{\ten}{\tensorsym}
\DeclareMathOperator{\grad}{\nabla}
\DeclareMathOperator{\dive}{\text{div}}
\DeclareMathOperator{\curl}{\text{curl}}
\newtheorem{remark}{Remark}
\newtheorem{definition}{Definition}
\newcommand{\R}{\mathbb{R}}
\newcommand{\D}{\mathcal{D}}

\newcommand{\tin}{\text{in}}
\newcommand{\ton}{\text{on}}

\newtheorem{theorem}{Theorem}
\newtheorem{lemma}{Lemma}

\begin{document}

\maketitle

\begin{itemize}
    \item Modifique la demostración de existencia y unicidad para problemas de punto silla para verificar el siguiente resultado: Sean $H_i, Q_i$ espacios de Hilbert sobre $\R$ con $i\in\{1,2\}$, y considere las formas bilineales acotadas $a:H_1\times H_2\to \R$ y $b_i:H_i\times Q_i\to \R$ para $i\in\{1,2\}$ con operadores inducidos $\mat A\in L(H_1, H_2)$ y $B_i\in L(H_i,Q_i)$ respectivamente. Definimos como $K_i$ el espacio nulo de $B_i$, y $\Pi_2$ el proyector ortogonal de $H_2$ en $K_2$. Suponga que:
        \begin{itemize}
            \item $\Pi_2\mat A:K_1\to K_2$ es un isomorfismo
            \item existen dos constantes $\beta_i$ positivas tales que
                $$ \sup_{v\in H_i} \frac{b_i(v,q)}{\|v\|_{H_i}} \geq \beta_i \| q \|_{Q_i} \qquad \forall q \in Q_i, \quad i \in \{1,2\}. $$
        \end{itemize}
        Pruebe que dados $F\in H_2'$ y $G\in Q_1'$, existe un único $(u,p)$
    \item Asuma que el operador
        $$ M = \begin{bmatrix} A & B^T \\ B & 0 \end{bmatrix} $$
            satisface todas las hipótesis de la teoría de Babuska-Brezzi, con formas bilineales $a$ y $b$. Defina precisamente el operador asociado a la siguiente forma bilineal:
            $$ l( (u,p), (v,q)) = a(u,v) + b(u,q) + b(v,p) $$
            junto a sus espacios funcionales, y demuestre que se satisfacen las condiciones inf-sup de sobreyectividad e inyectividad para $\mathcal L$,el operador inducido por $l$ .
    \item En clases vimos la formulación primal-mixta del Laplaciano. Demuestre usando la teoría de formulaciones de punto silla la existencia y unicidad de soluciones. Busque además espacios discretos que satisfagan la condición inf-sup discreta para proponer un esquema de Galerkin que sea estable y convergente.

    \item Dada una solución analítica, calcule las tasas numéricas de convergencia del par inf-sup estable $\mathbb{RT_k}\times \P_k$ para el problema de Darcy, y luego repita el ejercicio con el espacio de \emph{Taylor-Hood} $[\P_{k+1}]^d\times \P_k$. Muestre cómo cambia la convergencia si considera un par de espacios que no sean inf-sup estables, como por ejemplo $[\P_k]^d\times \P_k$, y grafique las soluciones para ver la diferencia entre ellas.
    \item Busque una demostración de la condición inf-sup continua del operador divergencia como un operador de $H^1$ a $L^2$. Puede consultar los libros de Brenner y Scott o de Girault y Raviart.
    \item Considere una función $\vec f$ en $H^{-1}(\Omega)$ y un dominio Lipschitz $\Omega$. Muestre que, para soluciones lo suficientemente pequeñas, existe al menos una solución para el siguiente problema no-lineal: Encontrar $(\vec u,p)$ en $\vec H_0^1(\Omega)\times L_0^2(\Omega)$ tal que
        $$
        \begin{aligned}
            -\mu\Delta \vec u + (\grad \vec u)\vec u + \grad p &= \vec f &&\qquad \Omega,\\
            \dive \vec u &= 0 && \qquad\Omega,
        \end{aligned}
        $$
        donde $L_0^2(\Omega) = \{q\in L^2(\Omega): \int_\Omega q = 0\}$. El sentido de "soluciones suficientemente pequeñas" dependerá del enfoque que use para la demostración. Si usa un resultado de invertibilidad local, entonces tendrá relación a la cercanía de la solución con respecto al punto de linealización. Si en cambio usa un resultado de punto fijo, tendrá que ver con el dominio en el que está definido el operador de Picard. Si busca usar un enfoque de punto fijo (ojalá si), entonces se sugiere estudiar primero el problema de Oseen, que es idéntico al mostrado antes pero requiere una función $\vec w$ dada:
        $$
        \begin{aligned}
            -\mu\Delta \vec u + (\grad \vec u)\vec w + \grad p &= \vec f &&\qquad \Omega,\\
            \dive \vec u &= 0 && \qquad\Omega.
        \end{aligned}
        $$
        Podrá encontrar en la literatura varias estrategias para resolver este ejercicio. Si logra demostrar la existencia de soluciones bajo ninguna hipótesis extra más allá del enunciado del modelo, \emph{debe notificar al docente del curso de manera totalmente urgente e inmediata} :). La notación clásica de $(\grad \vec u)\vec w$ es $\vec w\cdot \grad \vec u$. Use la que prefiera.


    \item Considere el siguiente problema variacional conocido como el p-Laplaciano:
            $$ \min_{u \in W_0^{1,p}(\Omega)} \frac 1 p \int_\Omega |\grad u|^p\,dx. $$
            Para este problema, (1) encuentre el problema distribucional asociado a sus ecuaciones de Euler-Lagrange (condiciones de primer orden), y (2) muestre la existencia de al menos una solución usando la teoría de Minty-Browder. Se sugiere revisar las secciones relevantes del libro de Ciarlet.
    \item Proponga un proyecto de curso. Para esto, proponga un modelo de estudio para el cual debería demostrar existencia de soluciones, unicidad si corresponde y además proponer un esquema de elementos finitos convergente. Si esto no es posible, deberá dar una respuesta detallada de por qué esto no es posible, y al menos reportar un caso simplificado en que si se podría. 
\end{itemize}
\end{document}

